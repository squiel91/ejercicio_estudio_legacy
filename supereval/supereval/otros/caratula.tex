\documentclass[12pt,a4paper]{article}
\usepackage[spanish]{babel}
\usepackage[latin1]{inputenc}

\usepackage{graphicx}
\usepackage{pstricks-add}
\usepackage{multicol}
\usepackage{array}
\usepackage{multirow}


\hoffset = -.5in \marginparwidth = 0cm \evensidemargin = 0cm
\marginparpush = 0cm

\voffset = -1in

\textwidth 17cm \textheight 25cm


\usepackage{amssymb}
\usepackage{amsmath}
\usepackage{amsfonts}

\newcounter{figura}
\renewcommand*{\thefigura}{\arabic{figura}}


\newcommand{\columnas}{\mathrm{col}}
\newcommand{\imagen}{\mathrm{im}}
\newcommand{\filas}{\mathrm{fil}}
\newcommand{\coord}{\mathrm{coord}}
\newcommand{\rango}{\mathrm{r}}
\newcommand{\sda}{\hspace{14.8cm}\phantom{.}} %Sangr�a derecha de 14cm para Opciones
\newcommand{\sdb}{\hspace{14.5cm}\phantom{.}} %Sangr�a derecha de 13cm para Opciones
\newcommand{\sdc}{\hspace{14.0cm}\phantom{.}} %Sangr�a derecha de 12cm para Opciones
\newcommand{\sdd}{\hspace{13.5cm}\phantom{.}} %Sangr�a derecha de 11cm para Opciones
\newcommand{\sde}{\hspace{13.0cm}\phantom{.}} %Sangr�a derecha de 11cm para Opciones

\renewcommand{\sin}{\, \mathrm{sen}}

\renewcommand{\theenumi}{\Alph{enumi}}
\begin{document}
\pagestyle{plain}


\newtheorem {Ejercicio}{Ejercicio}
\newenvironment{ejercicio}{\begin{Ejercicio} \rm}{\rm\end{Ejercicio}}
\newtheorem {Quiz}[Ejercicio]{Ejercicio *}
\newenvironment{quiz}{\begin{Quiz} \rm}{\rm\end{Quiz}}
\newtheorem{Lectura}{Lectura}
\newenvironment{lectura}{\begin{lectura} \rm}{\rm\end{lectura}}

\newtheorem {Pregunta}{Pregunta}
\newenvironment{pregunta}{\begin{Pregunta} \rm}{\rm\end{Pregunta}}

\newenvironment{opciones}{\begin{enumerate}\renewcommand{\theenumi}{\Alph{enumi}}}
{\end{enumerate}\renewcommand{\theenumi}{\roman{enumi}}}


\newtheorem{proposicion}{Proposici�n}

%%%%%%%%%%%%%%%%%%%%%%%%%%%%%%%%%%%%%%%%%%%%%%%%%%%%%%%%%%%%%%%%%%%%%%%%%%%%%%%%%%%%%%%%%%%%%
%%                       COMANDOS PARA LA CARATULA                                         %% 

\newcommand{\semestre}{2016 - Primer semestre}   %a�o y semestre en que se toma la prueba
\newcommand{\prueba}{PARCIAL 3}           % tipo y numero de prueba
\newcommand{\fecha}{9 de julio de 2016}   % fecha de la prueba
\newcommand{\anterioraprimerapregunta}{0} % fija la numeracion. Hay que poner aqui el n�mero anterior al de la primera pregunta

%\begin{center}
%\textbf{Nro. de asiento} \\
%\framebox{\rule{0cm}{2cm}\rule{3cm}{0cm}}
%\end{center}
%\medskip

\hrule

\medskip

\noindent \textbf{C�tedra de Matem�tica}
 \hfill \textbf{Facultad de Arquitectura, Dise�o y Urbanismo}

\hfill \textbf{Universidad de la Rep�blica}

\smallskip

\begin{center}
\noindent \large\bf{Matem�tica}
 \hfill \textbf{\semestre}

\medskip
\hrule \bigskip

 {\large\sc  \prueba -- \fecha}

\end{center}


\hrule
\bigskip

\noindent
\begin{center}
\begin{tabular}{|c|cc|cc}
\cline{1-1}
Nro. Asiento & & Grupo & C�dula: &
\\
\cline{1-1} \cline{5-5}
\rule{0cm}{.8cm}& & & &
\\
\rule{3cm}{0cm} &   & Equipo & Apellidos: &
\\
 \cline{5-5}
\rule{0cm}{.8cm} & & \rule{3cm}{0cm}& &
\\
 &  &  & Nombres: & \rule{6cm}{0cm}
\\
\cline{1-1} \cline{3-3} \cline{5-5}
\end{tabular}
\end{center}
\bigskip



\noindent \begin{center} {\large \bf TABLA DE RESPUESTAS}

\medskip

\textbf{\begin{tabular}{l|c|c|c|c|c|c|c|c|c|c|c}
Pregunta & \ref{P1} & \ref{P2} & \ref{P3} & \ref{P4} & \ref{P5} & \ref{P6} & \ref{P7}& \ref{P8} & \ref{P9} & \ref{P10}
\\
Respuesta \rule[-.4cm]{0cm}{1cm}  & \rule{.6cm}{0cm} &
\rule{.6cm}{0cm} & \rule{.6cm}{0cm} & \rule{.6cm}{0cm} &
\rule{.6cm}{0cm} & \rule{.6cm}{0cm} & \rule{.6cm}{0cm} &
\rule{.6cm}{0cm} & \rule{.6cm}{0cm} & \rule{.6cm}{0cm} &
\end{tabular}}
\end{center}

\medskip
\hrule
\medskip

\noindent \textbf{Instrucciones:}
\begin{itemize}
\item
Para cada pregunta que decidas contestar:
\begin{itemize}
\item
Colocar la letra de la opci�n seleccionada en la TABLA DE
RESPUESTAS. \textbf{S�lo tomaremos en cuenta las respuestas marcadas
en la tabla. Recuerda poner aqu� TODAS las respuestas a las
preguntas que quieras contestar}.
\item
Transcribir una s�ntesis de tu trabajo al espacio reservado (te
recomendamos utilizar esta instancia de resumir para repasar y
verificar el trabajo hecho). \textbf{S�lo se tendr�n en cuenta
respuestas a preguntas que est�n acompa�adas en el espacio
correspondiente de una argumentaci�n que justifique la opci�n
seleccionada}.
\end{itemize}
\item
Cada pregunta tiene una �nica opci�n correcta.
\item
Todas las preguntas tienen igual valor.
\item
Durante el parcial podr�s consultar material de apoyo y usar
calculadoras, de uso estrictamente personal.
%\item
\item
Esta instancia de evaluaci�n es estrictamente individual.
%Las respuestas equivocadas tienen una penalizaci�n del 15\%\ del valor de la pregunta.
%\item
%Los resultados ser�n publicados en la p�gina web de la
%C�tedra.
\item \textbf{Copia y guarda tus respuestas}.
\item Te recomendamos que trabajes en tu cuaderno, manteniendo registros ordenados de lo que
hiciste durante la prueba. La C\'atedra har\'a devoluciones sobre este trabajo
 y deber�s volver sobre �l si deseas acceder a la \textbf{recuperaci�n}.
%\textbf{Para la recuperaci�n de este parcial seremos m�s estrictos en el control de la revisi�n del trabajo hecho durante el parcial.}
%\item
\end{itemize}
\hrule



\pagebreak


%%%%%%%%%%%%%%%%%%%%%%%%%%%%%%%%%%%%%%%%%%%%%%%%

\setcounter{Pregunta}{\anterioraprimerapregunta}

%%%%%%%%%%%%%%%%%%  Pregunta 1   %%%%%%%%%%%%%%%%%%
\hrule
%\noindent

\begin{pregunta}
\label{P1}


ENUNCIADO DE LA PREGUNTA

\begin{opciones}
\item	
\item	
\item	
\item	
\end{opciones}
\vfill
\hfill
\framebox{\rule{0cm}{1cm}\rule{1cm}{0cm}}
\end{pregunta}
\hrule


%%%%%%%%%%%%%%%%%%  Pregunta 2   %%%%%%%%%%%%%%%%%%



\begin{pregunta}
\label{P2}

ENUNCIADO DE LA PREGUNTA

\begin{opciones}
\item	
\item	
\item	
\item	
\end{opciones}
\vfill
\hfill
\framebox{\rule{0cm}{1cm}\rule{1cm}{0cm}}
\end{pregunta}
\hrule



\pagebreak


%%%%%%%%%%%%%%%%%%  Pregunta 3   %%%%%%%%%%%%%%%%%%
\hrule

\begin{pregunta}
\label{P3}

ENUNCIADO DE LA PREGUNTA
\begin{opciones}
\item	
\item	
\item	
\item	
\end{opciones}
\vfill
\hfill
\framebox{\rule{0cm}{1cm}\rule{1cm}{0cm}}
\end{pregunta}
\hrule

%%%%%%%%%%%%%%%%%%  Pregunta 4   %%%%%%%%%%%%%%%%%%


\begin{pregunta}
\label{P4}


ENUNCIADO DE LA PREGUNTA

\begin{opciones}
\item	
\item	
\item	
\item	
\end{opciones}
\vfill
\hfill
\framebox{\rule{0cm}{1cm}\rule{1cm}{0cm}}
\end{pregunta}

\hrule

\pagebreak

\hrule
%%%%%%%%%%%%%%%%%%  Pregunta 5   %%%%%%%%%%%%%%%%%%

\begin{pregunta}
\label{P5}


ENUNCIADO DE LA PREGUNTA

\begin{opciones}
\item	
\item	
\item	
\item	
\end{opciones}
\vfill
\hfill
\framebox{\rule{0cm}{1cm}\rule{1cm}{0cm}}
\end{pregunta}
\hrule


%%%%%%%%%%%%%%%%%%  Pregunta 6   %%%%%%%%%%%%%%%%%%
\hrule
%\noindent
\begin{pregunta}
\label{P6}


ENUNCIADO DE LA PREGUNTA

\begin{opciones}
\item	
\item	
\item	
\item	
\end{opciones}
\vfill
\hfill
\framebox{\rule{0cm}{1cm}\rule{1cm}{0cm}}
\end{pregunta}
\hrule

\pagebreak
%%%%%%%%%%%%%%%%%%  Pregunta 7   %%%%%%%%%%%%%%%%%%

%\noindent
%\rule{17cm}{0.1cm}


\hrule
\begin{pregunta}
\label{P7}


ENUNCIADO DE LA PREGUNTA

\begin{opciones}
\item	
\item	
\item	
\item	
\end{opciones}
\vfill
\hfill
\framebox{\rule{0cm}{1cm}\rule{1cm}{0cm}}
\end{pregunta}

\hrule

%%%%%%%%%%%%%%%%%%  Pregunta 8   %%%%%%%%%%%%%%%%%%

%\noindent
\begin{pregunta}
\label{P8}


ENUNCIADO DE LA PREGUNTA

\begin{opciones}
\item	
\item	
\item	
\item	
\end{opciones}
\vfill
\hfill
\framebox{\rule{0cm}{1cm}\rule{1cm}{0cm}}
\end{pregunta}
\hrule


\pagebreak

%%%%%%%%%%%%%%%%%%  Pregunta 9   %%%%%%%%%%%%%%%%%%




\hrule

\begin{pregunta}
\label{P9}


ENUNCIADO DE LA PREGUNTA

\begin{opciones}
\item	
\item	
\item	
\item	
\end{opciones}
\vfill
\hfill
\framebox{\rule{0cm}{1cm}\rule{1cm}{0cm}}
\end{pregunta}
\hrule


%%%%%%%%%%%%%%%%%%  Pregunta 10   %%%%%%%%%%%%%%%%%%

%\noindent
\begin{pregunta}
\label{P10}


ENUNCIADO DE LA PREGUNTA

\begin{opciones}
\item	
\item	
\item	
\item	
\end{opciones}
\vfill
\hfill
\framebox{\rule{0cm}{1cm}\rule{1cm}{0cm}}
\end{pregunta}
\hrule



\end{document} 